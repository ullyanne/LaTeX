\documentclass[12 pt, a4 paper]{article}
\usepackage[utf8]{inputenc}
\usepackage[brazil]{babel}
\usepackage[lmargin=3cm,tmargin=3cm,rmargin=2cm,bmargin=2cm]{geometry}
\usepackage[T1]{fontenc}
\usepackage{amsmath}

\title{Quádricas}
\date{}

\begin{document}

\maketitle

\newpage

\tableofcontents

\newpage

\section{Introdução}

\textbf{Quádrica} ou superfície quádrica é, em matemática, o conjunto dos pontos do espaço tridimensional cujas coordenadas formam um polinômio de segundo grau de no máximo três variáveis denominada de equação cartesiana da superfície: 
\begin{center}

\begin{equation}
ax^2+by^2+cz^2+dxy+exz+fyz+gx+hy+iz+j=0
\end{equation}

\end{center}


\section{Superfícies}
\subsection{Elipsoide}
É uma superfície cuja equação num sistema de coordenadas cartesianas x-y-z é
\begin{center}

\begin{equation}
\dfrac{x^2}{a^2} + \dfrac{y^2}{b^2} + \dfrac{z^2}{c^2} = 1
\end{equation}

\end{center}

Onde a, b e c são números reais positivos que determinam as dimensões e forma do elipsoide. Se dois dos números são iguais, o elipsoide é um esferoide; se os três forem iguais, trata-se de uma esfera.

\subsection{Hiperboloide}
 Entre as superfícies quádricas, um hiperboloide é caracterizado por não ser um cone ou um cilindro, ter um centro de simetria e interceptar muitos planos em hipérboles. Um hiperboloide também possui três eixos perpendiculares de simetria emparelhados e três planos perpendiculares de simetria emparelhados.
 
\subsubsection{Hiperboloide de uma folha}
Hiperboloide de uma folha, também chamado de hiperboloide hiperbólico, é uma superfície conectada, que tem uma Curvatura Gaussiana negativa em cada ponto. Isto implica que o plano tangente em qualquer ponto intercepta o hiperboloide em duas retas.
\\
É definido pela equação:
\begin{center}

\begin{equation}
\dfrac{x^2}{a^2} + \dfrac{y^2}{b^2} - \dfrac{z^2}{c^2} = 1
\end{equation}

\end{center}

\subsubsection{Hiperboloide de duas folhas}
Hiperboloide de duas folhas, também chamado de hiperboloide elíptico, tem dois componentes conectados e uma curvatura gaussiana positiva em cada ponto. Assim, a superfície é convexa no sentido de que o plano tangente em todos os pontos intercepta a superfície somente nesse ponto.
\\
É definido pela equação:
\begin{center}

\begin{equation}
-\dfrac{x^2}{a^2} + \dfrac{y^2}{b^2} - \dfrac{z^2}{c^2} = 1
\end{equation}

\end{center}

\subsection{Paraboloide}
Um paraboloide é uma superfície que possui exatamente um eixo de simetria e nenhum centro de simetria. O termo "paraboloide" deriva de "parábola", que se refere a uma seção cônica a qual possui uma propriedade semelhante de simetria. Toda seção plana de um parabolóide por um plano paralelo ao eixo de simetria é uma parábola.

\subsubsection{Paraboloide elíptico}
Um paraboloide elíptico, também chamado de paraboloide circular, é um paraboloide de revolução: uma superfície obtida através da rotação de uma parábola ao redor de seu eixo e pode possuir um ponto máximo ou mínimo. Este é o formato do refletor parabólico utilizado nos espelhos, antenas e objetos semelhantes.
\\É definido pela equação:
\begin{center}

\begin{equation}
\dfrac{x^2}{a^2} + \dfrac{y^2}{b^2} = cz
\end{equation}

\end{center}

\subsubsection{Paraboloide hiperbólico}
O parabolóide é hiperbólico se qualquer outra seção do plano for uma hipérbole ou duas retas se cruzando. A superfície é duplamente determinada em forma de sela.
\\É definido pela equação:
\begin{center}

\begin{equation}
\dfrac{y^2}{b^2} - \dfrac{x^2}{a^2} = cz
\end{equation}

\end{center}

\subsection{Superfície cônica}
Superfície cônica é uma superfície gerada por uma reta que se move apoiada numa curva plana e passando sempre por um ponto dado não situado no plano desta curva. A reta é denominada geratriz, a curva plana é a diretriz e o ponto fixo dado é o vértice.
\\É definida pela equação:
\begin{center}

\begin{equation}
\dfrac{x^2}{a^2} + \dfrac{y^2}{b^2} - \dfrac{z^2}{c^2} = 0    
\end{equation}

\end{center}

\subsection{Superfície cilíndrica}
Uma superfície é dita cilíndrica se existir uma curva C e uma reta r tais que a superfície seja a união de retas paralelas a r que passem por C. C é chamada diretriz da superfície S e as retas paralelas a r são geratrizes de S.
Se a curva C for uma quádrica plana, então a superfície será uma quádrica no espaço.
\\A equação da superfície cilíndrica é a mesma de sua diretriz.

\end{document}
